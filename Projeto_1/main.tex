\documentclass[10pt]{article}
\usepackage[utf8]{inputenc}
\usepackage{caption}
\usepackage{graphicx}
\usepackage{array,tabularx}
\linespread{1.3} %spacing between lines. 1 for simple space, 1.3 for space and half, 1.6 for double
\usepackage[a4paper]{geometry}   %tipo de "página" com as margens a 2 cm
\usepackage{fancyhdr}  %isto é para os cabeçalhos e as imagens darem menos trabalho a colocar
\usepackage{hyperref}
\usepackage{array}
\usepackage{indentfirst}
\usepackage[portuguese]{babel}
\usepackage{amsmath}
\usepackage{listings} 
\usepackage{caption}
\usepackage{subcaption}
\usepackage{pdflscape}
\usepackage{tikz}
\usepackage{float}
\usepackage{multirow}
\usepackage{ctable}
\usepackage{ffcode}
\usepackage[toc,page]{appendix}
\usetikzlibrary{shapes.geometric, arrows}
\definecolor{mygreen}{rgb}{0,0.6,0}
\definecolor{mygray}{rgb}{0.5,0.5,0.5}
\definecolor{mymauve}{rgb}{0.58,0,0.82}
\graphicspath{{images/}} % Specifies the directory where pictures are stored

\makeatletter
\setlength{\@fptop}{0pt}
\makeatother

\hypersetup{
    colorlinks=true, %set true if you want colored links
    linktoc=all,     %set to all if you want both sections and subsections linked
    linkcolor=black,  %choose some color if you want links to stand out
    urlcolor=blue,
}
\DeclareGraphicsExtensions{.pdf,.png,.jpg}
\lstset{ %
    %backgroundcolor=\color{lightgray},   % choose the background color; you must add \usepackage{color} or \usepackage{xcolor}
    basicstyle=\footnotesize,        % the size of the fonts that are used for the code
    backgroundcolor=\color{black!5}, % set backgroundcolor
    breakatwhitespace=false,         % sets if automatic breaks should only happen at whitespace
    breaklines=true,                 % sets automatic line breaking
    captionpos=b,                    % sets the caption-position to bottom
    commentstyle=\color{mygreen},    % comment style
    deletekeywords={...},            % if you want to delete keywords from the given language
    escapeinside={\%*}{*)},          % if you want to add LaTeX within your code
    extendedchars=true,              % lets you use non-ASCII characters; for 8-bits encodings only, does not work with UTF-8
    frame=lines,                    % adds a frame around the code
    keepspaces=true,                 % keeps spaces in text, useful for keeping indentation of code (possibly needs columns=flexible)
    keywordstyle=\color{blue},       % keyword style
    language=C,                      % the language of the code
    morekeywords={*,...},            % if you want to add more keywords to the set
    numbers=none,                    % where to put the line-numbers; possible values are (none, left, right)
    numbersep=5pt,                   % how far the line-numbers are from the code
    numberstyle=\tiny\color{mygray}, % the style that is used for the line-numbers
    rulecolor=\color{black},         % if not set, the frame-color may be changed on line-breaks within not-black text (e.g. comments (green here))
    showspaces=false,                % show spaces everywhere adding particular underscores; it overrides 'showstringspaces'
    showstringspaces=false,          % underline spaces within strings only
    showtabs=false,                  % show tabs within strings adding particular underscores
    stepnumber=1,                    % the step between two line-numbers. If it's 1, each line will be numbered
    stringstyle=\color{mymauve},     % string literal style
    tabsize=2,                       % sets default tabsize to 2 spaces
    title=\lstname                   % show the filename of files included with \lstinputlisting; also try caption instead of title
  }

\usepackage{xcolor}
\usepackage{listings}
\definecolor{vgreen}{RGB}{104,180,104}
\definecolor{vblue}{RGB}{49,49,255}
\definecolor{vorange}{RGB}{255,143,102}

\lstdefinestyle{verilog-style}
{
    language=Verilog,
    basicstyle=\small\ttfamily,
    keywordstyle=\color{vblue},
    identifierstyle=\color{black},
    commentstyle=\color{vgreen},
    numbers=left,
    numberstyle=\tiny\color{black},
    numbersep=10pt,
    tabsize=8,
    moredelim=*[s][\colorIndex]{[}{]},
    literate=*{:}{:}1
}

\makeatletter
\newcommand*\@lbracket{[}
\newcommand*\@rbracket{]}
\newcommand*\@colon{:}
\newcommand*\colorIndex{%
    \edef\@temp{\the\lst@token}%
    \ifx\@temp\@lbracket \color{black}%
    \else\ifx\@temp\@rbracket \color{black}%
    \else\ifx\@temp\@colon \color{black}%
    \else \color{vorange}%
    \fi\fi\fi
}
\makeatother

\usepackage{trace}

% Blocks Style:
\tikzstyle{decision} = [diamond, draw, fill=blue!20, 
    text width=4.5em, text badly centered, node distance=3cm, inner sep=0pt]
\tikzstyle{block} = [rectangle, draw, fill=blue!20, 
    text width=5em, text centered, rounded corners, minimum height=4em]
\tikzstyle{line} = [draw, -latex']
\tikzstyle{cloud} = [draw, ellipse,fill=red!20, node distance=3cm,
    minimum height=2em]


\begin{document}

%capa
\begin{titlepage}


\newcommand{\HRule}{\rule{\linewidth}{0.5mm}} % Defines a new command for the horizontal lines, change thickness here

\center % Center everything on the page
 \vspace*{1cm}
%-------------------------------------------------------------------
%	HEADING SECTIONS
%-------------------------------------------------------------------

\textsc{\LARGE Instituto Superior Técnico}\\[1.5cm] % Name of your university/college
\textsc{\Large  Sensores e Atuadores}\\[0.5cm] % Major heading such as course name
% \textsc{\large Lab. 1 - Sinais e Sistemas}\\[0.5cm] % Minor heading such as course title

%-------------------------------------------------------------------
%	TITLE SECTION
%-------------------------------------------------------------------

\HRule \\[0.4cm]
%{ \huge \bfseries }\\[0.4cm] % Area
{ \LARGE \bfseries Transdutor Ultrassónico}\\[0.4cm] % Subarea
{ \large \bfseries Laboratório 2 - Relatório Formal}\\[0.4cm] % Part
\HRule \\[1.5cm]
 
%-------------------------------------------------------------------
%	AUTHOR SECTION
%-------------------------------------------------------------------

\begin{center}
\large\textbf{Grupo 5}\\
\large\textbf{Diogo Ralo - 96921}\\ 
\large\textbf{Diogo Luís - 96922}\\ 
\large\textbf{Victor Macedo - 105095}\\ 
\large\textbf{MEE - 2022/2023} 
\end{center}



\begin{minipage}{0.4\textwidth}
\begin{flushright} \large
\end{flushright}
\end{minipage}\\[.5cm]

\begin{center} \large
 
\end{center}
\vspace*{0.35cm}
%-------------------------------------------------------------------
%	DATE SECTION
%-------------------------------------------------------------------

{\large \today }\\[0.9cm] % Date, change the \today to a set date if you want to be precise

%-------------------------------------------------------------------
%	LOGO SECTION
%-------------------------------------------------------------------


\includegraphics[scale=0.5]{tecnico.png}\\[1cm]
   

%-------------------------------------------------------------------
%\vfill % Fill the rest of the page with whitespace
\end{titlepage}

\pagebreak

%Ìndice
\renewcommand{\contentsname}{Índice}
\tableofcontents
\pagebreak

%%%%%%%%%%%%%%%%%%%%%%%%%%%%%%%%%%%%%%%%%%%%%%%%%%%%%%%%%%%%%%%%%%
%////////////////////INTRODUÇÃO\\\\\\\\\\\\\\\\\\\\\\\\\\\\\\\\\\\
%%%%%%%%%%%%%%%%%%%%%%%%%%%%%%%%%%%%%%%%%%%%%%%%%%%%%%%%%%%%%%%%%%
\section{Introdução}

Este projeto tem como objetivo projetar e simular um gerador de impulsos que tem duas funcionalidades: uma que se resume a gerar um único impulso, outra que se relaciona com a geração contínua de impulsos de forma periódica.

%%%%%%%%%%%%%%%%%%%%%%%%%%%%%%%%%%%%%%%%%%%%%%%%%%%%%%%%%%%%%%%%%%
%//////////////////////////CÓDIGO\\\\\\\\\\\\\\\\\\\\\\\\\\\\\\\\\
%%%%%%%%%%%%%%%%%%%%%%%%%%%%%%%%%%%%%%%%%%%%%%%%%%%%%%%%%%%%%%%%%%
\section{Código \textit{Verilog}}

\subsection{Código para o módulo de gerador de impulsos}



\vspace{1cm}

\subsection{Código para o módulo de \textit{testbench}}



%%%%%%%%%%%%%%%%%%%%%%%%%%%%%%%%%%%%%%%%%%%%%%%%%%%%%%%%%%%%%%%%%%
%///////////////////////////Layout\\\\\\\\\\\\\\\\\\\\\\\\\\\\\\\\
%%%%%%%%%%%%%%%%%%%%%%%%%%%%%%%%%%%%%%%%%%%%%%%%%%%%%%%%%%%%%%%%%%
\section{Simulações}



%%%%%%%%%%%%%%%%%%%%%%%%%%%%%%%%%%%%%%%%%%%%%%%%%%%%%%%%%%%%%%%%%%
%///////////////////////Conclusão\\\\\\\\\\\\\\\\\\\\\\\\\\\\\\\\
%%%%%%%%%%%%%%%%%%%%%%%%%%%%%%%%%%%%%%%%%%%%%%%%%%%%%%%%%%%%%%%%%%
\section{Conclusões}



%%%%%%%%%%%%%%%%%%%%%%%%%%%%%%%%%%%%%%%%%%%%%%%%%%%%%%%%%%%%%%%%%%
%///////////////////////Referências\\\\\\\\\\\\\\\\\\\\\\\\\\\\\\
%%%%%%%%%%%%%%%%%%%%%%%%%%%%%%%%%%%%%%%%%%%%%%%%%%%%%%%%%%%%%%%%%%



%%%%%%%%%%%%%%%%%%%%%%%%%%%%%%%%%%%%%%%%%%%%%%%%%%%%%%%%%%%%%%%%%%
%///////////////////////Anexos\\\\\\\\\\\\\\\\\\\\\\\\\\\\\\\
%%%%%%%%%%%%%%%%%%%%%%%%%%%%%%%%%%%%%%%%%%%%%%%%%%%%%%%%%%%%%%%%%%



% References %%%%
%\begin{thebibliography}{15}

%\bibitem{ref_que_usas_no_texto}Autores \emph{titulo} % se for url \url{www.qualquercoisa.pt}
% \cite{ref_que_usas_no_texto} para citar

%\end{thebibliography}

%Annex
%\pagebreak
%\begin{appendix}
%\section*{Appendix: Algorithm flowchart}
%COLOCAR IMAGENS
%\begin{figure}[h]
%  \centering
%    \includegraphics[scale=0.8]{tecnico.png}  
%    \caption{Legenda da imagem}
%  \label{fig:flowchart}
  %para chamar a imagem é só colocar \ref{fig:flowchart}
%\end{figure}


%\end{appendix}
\end{document}

